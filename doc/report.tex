
%% bare_adv.tex
%% V1.4b
%% 2015/08/26
%% by Michael Shell
%% See: 
%% http://www.michaelshell.org/
%% for current contact information.
%%
%% This is a skeleton file demonstrating the advanced use of IEEEtran.cls
%% (requires IEEEtran.cls version 1.8b or later) with an IEEE Computer
%% Society journal paper.
%%
%% Support sites:
%% http://www.michaelshell.org/tex/ieeetran/
%% http://www.ctan.org/pkg/ieeetran
%% and
%% http://www.ieee.org/

%%*************************************************************************
%% Legal Notice:
%% This code is offered as-is without any warranty either expressed or
%% implied; without even the implied warranty of MERCHANTABILITY or
%% FITNESS FOR A PARTICULAR PURPOSE! 
%% User assumes all risk.
%% In no event shall the IEEE or any contributor to this code be liable for
%% any damages or losses, including, but not limited to, incidental,
%% consequential, or any other damages, resulting from the use or misuse
%% of any information contained here.
%%
%% All comments are the opinions of their respective authors and are not
%% necessarily endorsed by the IEEE.
%%
%% This work is distributed under the LaTeX Project Public License (LPPL)
%% ( http://www.latex-project.org/ ) version 1.3, and may be freely used,
%% distributed and modified. A copy of the LPPL, version 1.3, is included
%% in the base LaTeX documentation of all distributions of LaTeX released
%% 2003/12/01 or later.
%% Retain all contribution notices and credits.
%% ** Modified files should be clearly indicated as such, including  **
%% ** renaming them and changing author support contact information. **
%%*************************************************************************


% *** Authors should verify (and, if needed, correct) their LaTeX system  ***
% *** with the testflow diagnostic prior to trusting their LaTeX platform ***
% *** with production work. The IEEE's font choices and paper sizes can   ***
% *** trigger bugs that do not appear when using other class files.       ***                          ***
% The testflow support page is at:
% http://www.michaelshell.org/tex/testflow/


% IEEEtran V1.7 and later provides for these CLASSINPUT macros to allow the
% user to reprogram some IEEEtran.cls defaults if needed. These settings
% override the internal defaults of IEEEtran.cls regardless of which class
% options are used. Do not use these unless you have good reason to do so as
% they can result in nonIEEE compliant documents. User beware. ;)
%
%\newcommand{\CLASSINPUTbaselinestretch}{1.0} % baselinestretch
%\newcommand{\CLASSINPUTinnersidemargin}{1in} % inner side margin
%\newcommand{\CLASSINPUToutersidemargin}{1in} % outer side margin
%\newcommand{\CLASSINPUTtoptextmargin}{1in}   % top text margin
%\newcommand{\CLASSINPUTbottomtextmargin}{1in}% bottom text margin




%
\documentclass[10pt,journal,compsoc]{IEEEtran}
% If IEEEtran.cls has not been installed into the LaTeX system files,
% manually specify the path to it like:
% \documentclass[10pt,journal,compsoc]{../sty/IEEEtran}


% For Computer Society journals, IEEEtran defaults to the use of 
% Palatino/Palladio as is done in IEEE Computer Society journals.
% To go back to Times Roman, you can use this code:
%\renewcommand{\rmdefault}{ptm}\selectfont


%\usepackage[polish]{babel}
\usepackage[table]{xcolor}
\usepackage{listings}


% Some very useful LaTeX packages include:
% (uncomment the ones you want to load)



% *** MISC UTILITY PACKAGES ***
%
%\usepackage{ifpdf}
% Heiko Oberdiek's ifpdf.sty is very useful if you need conditional
% compilation based on whether the output is pdf or dvi.
% usage:
% \ifpdf
%   % pdf code
% \else
%   % dvi code
% \fi
% The latest version of ifpdf.sty can be obtained from:
% http://www.ctan.org/pkg/ifpdf
% Also, note that IEEEtran.cls V1.7 and later provides a builtin
% \ifCLASSINFOpdf conditional that works the same way.
% When switching from latex to pdflatex and vice-versa, the compiler may
% have to be run twice to clear warning/error messages.






% *** CITATION PACKAGES ***
%
\ifCLASSOPTIONcompsoc
  % The IEEE Computer Society needs nocompress option
  % requires cite.sty v4.0 or later (November 2003)
  \usepackage[nocompress]{cite}
\else
  % normal IEEE
  \usepackage{cite}
\fi
% cite.sty was written by Donald Arseneau
% V1.6 and later of IEEEtran pre-defines the format of the cite.sty package
% \cite{} output to follow that of the IEEE. Loading the cite package will
% result in citation numbers being automatically sorted and properly
% "compressed/ranged". e.g., [1], [9], [2], [7], [5], [6] without using
% cite.sty will become [1], [2], [5]--[7], [9] using cite.sty. cite.sty's
% \cite will automatically add leading space, if needed. Use cite.sty's
% noadjust option (cite.sty V3.8 and later) if you want to turn this off
% such as if a citation ever needs to be enclosed in parenthesis.
% cite.sty is already installed on most LaTeX systems. Be sure and use
% version 5.0 (2009-03-20) and later if using hyperref.sty.
% The latest version can be obtained at:
% http://www.ctan.org/pkg/cite
% The documentation is contained in the cite.sty file itself.
%
% Note that some packages require special options to format as the Computer
% Society requires. In particular, Computer Society  papers do not use
% compressed citation ranges as is done in typical IEEE papers
% (e.g., [1]-[4]). Instead, they list every citation separately in order
% (e.g., [1], [2], [3], [4]). To get the latter we need to load the cite
% package with the nocompress option which is supported by cite.sty v4.0
% and later.





% *** GRAPHICS RELATED PACKAGES ***
%
\ifCLASSINFOpdf
   \usepackage[pdftex]{graphicx}
  % declare the path(s) where your graphic files are
  % \graphicspath{{../pdf/}{../jpeg/}}
  % and their extensions so you won't have to specify these with
  % every instance of \includegraphics
  % \DeclareGraphicsExtensions{.pdf,.jpeg,.png}
\else
  % or other class option (dvipsone, dvipdf, if not using dvips). graphicx
  % will default to the driver specified in the system graphics.cfg if no
  % driver is specified.
  % \usepackage[dvips]{graphicx}
  % declare the path(s) where your graphic files are
  % \graphicspath{{../eps/}}
  % and their extensions so you won't have to specify these with
  % every instance of \includegraphics
  % \DeclareGraphicsExtensions{.eps}
\fi
% graphicx was written by David Carlisle and Sebastian Rahtz. It is
% required if you want graphics, photos, etc. graphicx.sty is already
% installed on most LaTeX systems. The latest version and documentation
% can be obtained at: 
% http://www.ctan.org/pkg/graphicx
% Another good source of documentation is "Using Imported Graphics in
% LaTeX2e" by Keith Reckdahl which can be found at:
% http://www.ctan.org/pkg/epslatex
%
% latex, and pdflatex in dvi mode, support graphics in encapsulated
% postscript (.eps) format. pdflatex in pdf mode supports graphics
% in .pdf, .jpeg, .png and .mps (metapost) formats. Users should ensure
% that all non-photo figures use a vector format (.eps, .pdf, .mps) and
% not a bitmapped formats (.jpeg, .png). The IEEE frowns on bitmapped formats
% which can result in "jaggedy"/blurry rendering of lines and letters as
% well as large increases in file sizes.
%
% You can find documentation about the pdfTeX application at:
% http://www.tug.org/applications/pdftex





% *** MATH PACKAGES ***
%
%\usepackage{amsmath}
% A popular package from the American Mathematical Society that provides
% many useful and powerful commands for dealing with mathematics.
%
% Note that the amsmath package sets \interdisplaylinepenalty to 10000
% thus preventing page breaks from occurring within multiline equations. Use:
%\interdisplaylinepenalty=2500
% after loading amsmath to restore such page breaks as IEEEtran.cls normally
% does. amsmath.sty is already installed on most LaTeX systems. The latest
% version and documentation can be obtained at:
% http://www.ctan.org/pkg/amsmath





% *** SPECIALIZED LIST PACKAGES ***
%\usepackage{acronym}
% acronym.sty was written by Tobias Oetiker. This package provides tools for
% managing documents with large numbers of acronyms. (You don't *have* to
% use this package - unless you have a lot of acronyms, you may feel that
% such package management of them is bit of an overkill.)
% Do note that the acronym environment (which lists acronyms) will have a
% problem when used under IEEEtran.cls because acronym.sty relies on the
% description list environment - which IEEEtran.cls has customized for
% producing IEEE style lists. A workaround is to declared the longest
% label width via the IEEEtran.cls \IEEEiedlistdecl global control:
%
% \renewcommand{\IEEEiedlistdecl}{\IEEEsetlabelwidth{SONET}}
% \begin{acronym}
%
% \end{acronym}
% \renewcommand{\IEEEiedlistdecl}{\relax}% remember to reset \IEEEiedlistdecl
%
% instead of using the acronym environment's optional argument.
% The latest version and documentation can be obtained at:
% http://www.ctan.org/pkg/acronym


%\usepackage{algorithmic}
% algorithmic.sty was written by Peter Williams and Rogerio Brito.
% This package provides an algorithmic environment fo describing algorithms.
% You can use the algorithmic environment in-text or within a figure
% environment to provide for a floating algorithm. Do NOT use the algorithm
% floating environment provided by algorithm.sty (by the same authors) or
% algorithm2e.sty (by Christophe Fiorio) as the IEEE does not use dedicated
% algorithm float types and packages that provide these will not provide
% correct IEEE style captions. The latest version and documentation of
% algorithmic.sty can be obtained at:
% http://www.ctan.org/pkg/algorithms
% Also of interest may be the (relatively newer and more customizable)
% algorithmicx.sty package by Szasz Janos:
% http://www.ctan.org/pkg/algorithmicx




% *** ALIGNMENT PACKAGES ***
%
%\usepackage{array}
% Frank Mittelbach's and David Carlisle's array.sty patches and improves
% the standard LaTeX2e array and tabular environments to provide better
% appearance and additional user controls. As the default LaTeX2e table
% generation code is lacking to the point of almost being broken with
% respect to the quality of the end results, all users are strongly
% advised to use an enhanced (at the very least that provided by array.sty)
% set of table tools. array.sty is already installed on most systems. The
% latest version and documentation can be obtained at:
% http://www.ctan.org/pkg/array


%\usepackage{mdwmath}
%\usepackage{mdwtab}
% Also highly recommended is Mark Wooding's extremely powerful MDW tools,
% especially mdwmath.sty and mdwtab.sty which are used to format equations
% and tables, respectively. The MDWtools set is already installed on most
% LaTeX systems. The lastest version and documentation is available at:
% http://www.ctan.org/pkg/mdwtools


% IEEEtran contains the IEEEeqnarray family of commands that can be used to
% generate multiline equations as well as matrices, tables, etc., of high
% quality.


%\usepackage{eqparbox}
% Also of notable interest is Scott Pakin's eqparbox package for creating
% (automatically sized) equal width boxes - aka "natural width parboxes".
% Available at:
% http://www.ctan.org/pkg/eqparbox




% *** SUBFIGURE PACKAGES ***
%\ifCLASSOPTIONcompsoc
%  \usepackage[caption=false,font=footnotesize,labelfont=sf,textfont=sf]{subfig}
%\else
%  \usepackage[caption=false,font=footnotesize]{subfig}
%\fi
% subfig.sty, written by Steven Douglas Cochran, is the modern replacement
% for subfigure.sty, the latter of which is no longer maintained and is
% incompatible with some LaTeX packages including fixltx2e. However,
% subfig.sty requires and automatically loads Axel Sommerfeldt's caption.sty
% which will override IEEEtran.cls' handling of captions and this will result
% in non-IEEE style figure/table captions. To prevent this problem, be sure
% and invoke subfig.sty's "caption=false" package option (available since
% subfig.sty version 1.3, 2005/06/28) as this is will preserve IEEEtran.cls
% handling of captions.
% Note that the Computer Society format requires a sans serif font rather
% than the serif font used in traditional IEEE formatting and thus the need
% to invoke different subfig.sty package options depending on whether
% compsoc mode has been enabled.
%
% The latest version and documentation of subfig.sty can be obtained at:
% http://www.ctan.org/pkg/subfig




% *** FLOAT PACKAGES ***
%
%\usepackage{fixltx2e}
% fixltx2e, the successor to the earlier fix2col.sty, was written by
% Frank Mittelbach and David Carlisle. This package corrects a few problems
% in the LaTeX2e kernel, the most notable of which is that in current
% LaTeX2e releases, the ordering of single and double column floats is not
% guaranteed to be preserved. Thus, an unpatched LaTeX2e can allow a
% single column figure to be placed prior to an earlier double column
% figure.
% Be aware that LaTeX2e kernels dated 2015 and later have fixltx2e.sty's
% corrections already built into the system in which case a warning will
% be issued if an attempt is made to load fixltx2e.sty as it is no longer
% needed.
% The latest version and documentation can be found at:
% http://www.ctan.org/pkg/fixltx2e


%\usepackage{stfloats}
% stfloats.sty was written by Sigitas Tolusis. This package gives LaTeX2e
% the ability to do double column floats at the bottom of the page as well
% as the top. (e.g., "\begin{figure*}[!b]" is not normally possible in
% LaTeX2e). It also provides a command:
%\fnbelowfloat
% to enable the placement of footnotes below bottom floats (the standard
% LaTeX2e kernel puts them above bottom floats). This is an invasive package
% which rewrites many portions of the LaTeX2e float routines. It may not work
% with other packages that modify the LaTeX2e float routines. The latest
% version and documentation can be obtained at:
% http://www.ctan.org/pkg/stfloats
% Do not use the stfloats baselinefloat ability as the IEEE does not allow
% \baselineskip to stretch. Authors submitting work to the IEEE should note
% that the IEEE rarely uses double column equations and that authors should try
% to avoid such use. Do not be tempted to use the cuted.sty or midfloat.sty
% packages (also by Sigitas Tolusis) as the IEEE does not format its papers in
% such ways.
% Do not attempt to use stfloats with fixltx2e as they are incompatible.
% Instead, use Morten Hogholm'a dblfloatfix which combines the features
% of both fixltx2e and stfloats:
%
% \usepackage{dblfloatfix}
% The latest version can be found at:
% http://www.ctan.org/pkg/dblfloatfix


%\ifCLASSOPTIONcaptionsoff
%  \usepackage[nomarkers]{endfloat}
% \let\MYoriglatexcaption\caption
% \renewcommand{\caption}[2][\relax]{\MYoriglatexcaption[#2]{#2}}
%\fi
% endfloat.sty was written by James Darrell McCauley, Jeff Goldberg and 
% Axel Sommerfeldt. This package may be useful when used in conjunction with 
% IEEEtran.cls'  captionsoff option. Some IEEE journals/societies require that
% submissions have lists of figures/tables at the end of the paper and that
% figures/tables without any captions are placed on a page by themselves at
% the end of the document. If needed, the draftcls IEEEtran class option or
% \CLASSINPUTbaselinestretch interface can be used to increase the line
% spacing as well. Be sure and use the nomarkers option of endfloat to
% prevent endfloat from "marking" where the figures would have been placed
% in the text. The two hack lines of code above are a slight modification of
% that suggested by in the endfloat docs (section 8.4.1) to ensure that
% the full captions always appear in the list of figures/tables - even if
% the user used the short optional argument of \caption[]{}.
% IEEE papers do not typically make use of \caption[]'s optional argument,
% so this should not be an issue. A similar trick can be used to disable
% captions of packages such as subfig.sty that lack options to turn off
% the subcaptions:
% For subfig.sty:
% \let\MYorigsubfloat\subfloat
% \renewcommand{\subfloat}[2][\relax]{\MYorigsubfloat[]{#2}}
% However, the above trick will not work if both optional arguments of
% the \subfloat command are used. Furthermore, there needs to be a
% description of each subfigure *somewhere* and endfloat does not add
% subfigure captions to its list of figures. Thus, the best approach is to
% avoid the use of subfigure captions (many IEEE journals avoid them anyway)
% and instead reference/explain all the subfigures within the main caption.
% The latest version of endfloat.sty and its documentation can obtained at:
% http://www.ctan.org/pkg/endfloat
%
% The IEEEtran \ifCLASSOPTIONcaptionsoff conditional can also be used
% later in the document, say, to conditionally put the References on a 
% page by themselves.





% *** PDF, URL AND HYPERLINK PACKAGES ***
%
%\usepackage{url}
% url.sty was written by Donald Arseneau. It provides better support for
% handling and breaking URLs. url.sty is already installed on most LaTeX
% systems. The latest version and documentation can be obtained at:
% http://www.ctan.org/pkg/url
% Basically, \url{my_url_here}.


% NOTE: PDF thumbnail features are not required in IEEE papers
%       and their use requires extra complexity and work.
%\ifCLASSINFOpdf
%  \usepackage[pdftex]{thumbpdf}
%\else
%  \usepackage[dvips]{thumbpdf}
%\fi
% thumbpdf.sty and its companion Perl utility were written by Heiko Oberdiek.
% It allows the user a way to produce PDF documents that contain fancy
% thumbnail images of each of the pages (which tools like acrobat reader can
% utilize). This is possible even when using dvi->ps->pdf workflow if the
% correct thumbpdf driver options are used. thumbpdf.sty incorporates the
% file containing the PDF thumbnail information (filename.tpm is used with
% dvips, filename.tpt is used with pdftex, where filename is the base name of
% your tex document) into the final ps or pdf output document. An external
% utility, the thumbpdf *Perl script* is needed to make these .tpm or .tpt
% thumbnail files from a .ps or .pdf version of the document (which obviously
% does not yet contain pdf thumbnails). Thus, one does a:
% 
% thumbpdf filename.pdf 
%
% to make a filename.tpt, and:
%
% thumbpdf --mode dvips filename.ps
%
% to make a filename.tpm which will then be loaded into the document by
% thumbpdf.sty the NEXT time the document is compiled (by pdflatex or
% latex->dvips->ps2pdf). Users must be careful to regenerate the .tpt and/or
% .tpm files if the main document changes and then to recompile the
% document to incorporate the revised thumbnails to ensure that thumbnails
% match the actual pages. It is easy to forget to do this!
% 
% Unix systems come with a Perl interpreter. However, MS Windows users
% will usually have to install a Perl interpreter so that the thumbpdf
% script can be run. The Ghostscript PS/PDF interpreter is also required.
% See the thumbpdf docs for details. The latest version and documentation
% can be obtained at.
% http://www.ctan.org/pkg/thumbpdf


% NOTE: PDF hyperlink and bookmark features are not required in IEEE
%       papers and their use requires extra complexity and work.
% *** IF USING HYPERREF BE SURE AND CHANGE THE EXAMPLE PDF ***
% *** TITLE/SUBJECT/AUTHOR/KEYWORDS INFO BELOW!!           ***
\newcommand\MYhyperrefoptions{bookmarks=true,bookmarksnumbered=true,
pdfpagemode={UseOutlines},plainpages=false,pdfpagelabels=true,
colorlinks=true,linkcolor={black},citecolor={black},urlcolor={black},
pdftitle={Bare Demo of IEEEtran.cls for Computer Society Journals},%<!CHANGE!
pdfsubject={Typesetting},%<!CHANGE!
pdfauthor={Michael D. Shell},%<!CHANGE!
pdfkeywords={Computer Society, IEEEtran, journal, LaTeX, paper,
             template}}%<^!CHANGE!
%\ifCLASSINFOpdf
%\usepackage[\MYhyperrefoptions,pdftex]{hyperref}
%\else
%\usepackage[\MYhyperrefoptions,breaklinks=true,dvips]{hyperref}
%\usepackage{breakurl}
%\fi
% One significant drawback of using hyperref under DVI output is that the
% LaTeX compiler cannot break URLs across lines or pages as can be done
% under pdfLaTeX's PDF output via the hyperref pdftex driver. This is
% probably the single most important capability distinction between the
% DVI and PDF output. Perhaps surprisingly, all the other PDF features
% (PDF bookmarks, thumbnails, etc.) can be preserved in
% .tex->.dvi->.ps->.pdf workflow if the respective packages/scripts are
% loaded/invoked with the correct driver options (dvips, etc.). 
% As most IEEE papers use URLs sparingly (mainly in the references), this
% may not be as big an issue as with other publications.
%
% That said, Vilar Camara Neto created his breakurl.sty package which
% permits hyperref to easily break URLs even in dvi mode.
% Note that breakurl, unlike most other packages, must be loaded
% AFTER hyperref. The latest version of breakurl and its documentation can
% be obtained at:
% http://www.ctan.org/pkg/breakurl
% breakurl.sty is not for use under pdflatex pdf mode.
%
% The advanced features offer by hyperref.sty are not required for IEEE
% submission, so users should weigh these features against the added
% complexity of use.
% The package options above demonstrate how to enable PDF bookmarks
% (a type of table of contents viewable in Acrobat Reader) as well as
% PDF document information (title, subject, author and keywords) that is
% viewable in Acrobat reader's Document_Properties menu. PDF document
% information is also used extensively to automate the cataloging of PDF
% documents. The above set of options ensures that hyperlinks will not be
% colored in the text and thus will not be visible in the printed page,
% but will be active on "mouse over". USING COLORS OR OTHER HIGHLIGHTING
% OF HYPERLINKS CAN RESULT IN DOCUMENT REJECTION BY THE IEEE, especially if
% these appear on the "printed" page. IF IN DOUBT, ASK THE RELEVANT
% SUBMISSION EDITOR. You may need to add the option hypertexnames=false if
% you used duplicate equation numbers, etc., but this should not be needed
% in normal IEEE work.
% The latest version of hyperref and its documentation can be obtained at:
% http://www.ctan.org/pkg/hyperref





% *** Do not adjust lengths that control margins, column widths, etc. ***
% *** Do not use packages that alter fonts (such as pslatex).         ***
% There should be no need to do such things with IEEEtran.cls V1.6 and later.
% (Unless specifically asked to do so by the journal or conference you plan
% to submit to, of course. )


% correct bad hyphenation here
\hyphenation{op-tical net-works semi-conduc-tor}


\begin{document}
%
% paper title
% Titles are generally capitalized except for words such as a, an, and, as,
% at, but, by, for, in, nor, of, on, or, the, to and up, which are usually
% not capitalized unless they are the first or last word of the title.
% Linebreaks \\ can be used within to get better formatting as desired.
% Do not put math or special symbols in the title.
\title{Emotion Recognition In Images From EEG\\
Signals Using Deep Neural Networks}
%
%
% author names and IEEE memberships
% note positions of commas and nonbreaking spaces ( ~ ) LaTeX will not break
% a structure at a ~ so this keeps an author's name from being broken across
% two lines.
% use \thanks{} to gain access to the first footnote area
% a separate \thanks must be used for each paragraph as LaTeX2e's \thanks
% was not built to handle multiple paragraphs
%
%
%\IEEEcompsocitemizethanks is a special \thanks that produces the bulleted
% lists the Computer Society journals use for "first footnote" author
% affiliations. Use \IEEEcompsocthanksitem which works much like \item
% for each affiliation group. When not in compsoc mode,
% \IEEEcompsocitemizethanks becomes like \thanks and
% \IEEEcompsocthanksitem becomes a line break with idention. This
% facilitates dual compilation, although admittedly the differences in the
% desired content of \author between the different types of papers makes a
% one-size-fits-all approach a daunting prospect. For instance, compsoc 
% journal papers have the author affiliations above the "Manuscript
% received ..."  text while in non-compsoc journals this is reversed. Sigh.

\author{Paweł~Figiel, Michał~Jakóbczyk, Kamil~Janas, Krzysztof~Lelonek and Dawid~Pieła% <-this % stops a space
\thanks{Manuscript received June 26, 2019.}}

% note the % following the last \IEEEmembership and also \thanks - 
% these prevent an unwanted space from occurring between the last author name
% and the end of the author line. i.e., if you had this:
% 
% \author{....lastname \thanks{...} \thanks{...} }
%                     ^------------^------------^----Do not want these spaces!
%
% a space would be appended to the last name and could cause every name on that
% line to be shifted left slightly. This is one of those "LaTeX things". For
% instance, "\textbf{A} \textbf{B}" will typeset as "A B" not "AB". To get
% "AB" then you have to do: "\textbf{A}\textbf{B}"
% \thanks is no different in this regard, so shield the last } of each \thanks
% that ends a line with a % and do not let a space in before the next \thanks.
% Spaces after \IEEEmembership other than the last one are OK (and needed) as
% you are supposed to have spaces between the names. For what it is worth,
% this is a minor point as most people would not even notice if the said evil
% space somehow managed to creep in.



% The paper headers
%\markboth{Journal of \LaTeX\ Class Files,~Vol.~14, No.~8, August~2015}%
%{Shell \MakeLowercase{\textit{et al.}}: Bare Advanced Demo of IEEEtran.cls for IEEE Computer Society Journals}
% The only time the second header will appear is for the odd numbered pages
% after the title page when using the twoside option.
% 
% *** Note that you probably will NOT want to include the author's ***
% *** name in the headers of peer review papers.                   ***
% You can use \ifCLASSOPTIONpeerreview for conditional compilation here if
% you desire.



% The publisher's ID mark at the bottom of the page is less important with
% Computer Society journal papers as those publications place the marks
% outside of the main text columns and, therefore, unlike regular IEEE
% journals, the available text space is not reduced by their presence.
% If you want to put a publisher's ID mark on the page you can do it like
% this:
%\IEEEpubid{0000--0000/00\$00.00~\copyright~2015 IEEE}
% or like this to get the Computer Society new two part style.
%\IEEEpubid{\makebox[\columnwidth]{\hfill 0000--0000/00/\$00.00~\copyright~2015 IEEE}%
%\hspace{\columnsep}\makebox[\columnwidth]{Published by the IEEE Computer Society\hfill}}
% Remember, if you use this you must call \IEEEpubidadjcol in the second
% column for its text to clear the IEEEpubid mark (Computer Society journal
% papers don't need this extra clearance.)



% use for special paper notices
%\IEEEspecialpapernotice{(Invited Paper)}



% for Computer Society papers, we must declare the abstract and index terms
% PRIOR to the title within the \IEEEtitleabstractindextext IEEEtran
% command as these need to go into the title area created by \maketitle.
% As a general rule, do not put math, special symbols or citations
% in the abstract or keywords.
\IEEEtitleabstractindextext{%
\begin{abstract}
Electroencephalography is a method of measuring electrical activity of brain. It can be used in many fields, as medicine or human-machine interface. But collecting data is just a first step, then they must be interpreted by some software. In this document there is described an experiment, which aim was to measure possibilities of EEG data analysis and emotion recognition using neural network. In this experiment there was used convolution neural network and EMOBRAIN database to train and test it. Results received were below 50\% accuracy, what could be caused mainly by low quality of used data. Experiment indicated that high quality and well preparing of data can have key impact for effectiveness of neural network and classification at all.
\end{abstract}

% Note that keywords are not normally used for peerreview papers.
%\begin{IEEEkeywords}
%Computer Society, IEEE, IEEEtran, journal, \LaTeX, paper, template.
%\end{IEEEkeywords}
}


% make the title area
\maketitle


% To allow for easy dual compilation without having to reenter the
% abstract/keywords data, the \IEEEtitleabstractindextext text will
% not be used in maketitle, but will appear (i.e., to be "transported")
% here as \IEEEdisplaynontitleabstractindextext when compsoc mode
% is not selected <OR> if conference mode is selected - because compsoc
% conference papers position the abstract like regular (non-compsoc)
% papers do!
\IEEEdisplaynontitleabstractindextext
% \IEEEdisplaynontitleabstractindextext has no effect when using
% compsoc under a non-conference mode.


% For peer review papers, you can put extra information on the cover
% page as needed:
% \ifCLASSOPTIONpeerreview
% \begin{center} \bfseries EDICS Category: 3-BBND \end{center}
% \fi
%
% For peerreview papers, this IEEEtran command inserts a page break and
% creates the second title. It will be ignored for other modes.
\IEEEpeerreviewmaketitle


\ifCLASSOPTIONcompsoc
\IEEEraisesectionheading{\section{Introduction}\label{sec:introduction}}
\else
\section{Introduction}
\label{sec:introduction}
\fi
% Computer Society journal (but not conference!) papers do something unusual
% with the very first section heading (almost always called "Introduction").
% They place it ABOVE the main text! IEEEtran.cls does not automatically do
% this for you, but you can achieve this effect with the provided
% \IEEEraisesectionheading{} command. Note the need to keep any \label that
% is to refer to the section immediately after \section in the above as
% \IEEEraisesectionheading puts \section within a raised box.




% The very first letter is a 2 line initial drop letter followed
% by the rest of the first word in caps (small caps for compsoc).
% 
% form to use if the first word consists of a single letter:
% \IEEEPARstart{A}{demo} file is ....
% 
% form to use if you need the single drop letter followed by
% normal text (unknown if ever used by the IEEE):
% \IEEEPARstart{A}{}demo file is ....
% 
% Some journals put the first two words in caps:
% \IEEEPARstart{T}{his demo} file is ....
% 
% Here we have the typical use of a "T" for an initial drop letter
% and "HIS" in caps to complete the first word.
\IEEEPARstart{L}{iving} brain generates electric field depending of its current activity, physical state, emotions, etc. Measuring that electrical brain activity enables "brain reading". It can be used in many fields, e.g. medicine (brain diseases diagnosis), criminology (lie detector), human-machine interfaces (especially for paralyzed people). Reading electric signals from brain is possible with using electroencephalography (EEG). Weakness of EEG is that the noninvasive variant is very sensitive for any noises, so it is practically useless in normal, not laboratorial, environment.

Important challenge for EEG is analysis and interpretation of collected data. One of possible approaches to this issue is using neural network to analysis of EEG data. The aim this project is an examination of  effectiveness this approach by making an experience of emotion recognition using EEG data and neural network. There will be designed and implemented convolution neural network. Data for training and testing the network will be received from EMOBRAIN database. This work uses Python to implementation of the research instrument. To implement the network there will be used Keras and Tensorflow libraries and for preprocessing and extraction data from EEG files there will be used MNE library.

% You must have at least 2 lines in the paragraph with the drop letter
% (should never be an issue)

%\hfill mds
 
%\hfill August 26, 2015

%\subsection{Subsection Heading Here}
%Subsection text here.

% needed in second column of first page if using \IEEEpubid
%\IEEEpubidadjcol

%\subsubsection{Subsubsection Heading Here}
%Subsubsection text here.


% An example of a floating figure using the graphicx package.
% Note that \label must occur AFTER (or within) \caption.
% For figures, \caption should occur after the \includegraphics.
% Note that IEEEtran v1.7 and later has special internal code that
% is designed to preserve the operation of \label within \caption
% even when the captionsoff option is in effect. However, because
% of issues like this, it may be the safest practice to put all your
% \label just after \caption rather than within \caption{}.
%
% Reminder: the "draftcls" or "draftclsnofoot", not "draft", class
% option should be used if it is desired that the figures are to be
% displayed while in draft mode.
%
%\begin{figure}[!t]
%\centering
%\includegraphics[width=2.5in]{mygraphic}
% where an .eps filename suffix will be assumed under latex, 
% and a .pdf suffix will be assumed for pdflatex; or what has been declared
% via \DeclareGraphicsExtensions.
%\caption{Simulation results for the network.}
%\label{fig_sim}
%\end{figure}

% Note that the IEEE typically puts floats only at the top, even when this
% results in a large percentage of a column being occupied by floats.
% However, the Computer Society has been known to put floats at the bottom.


% An example of a double column floating figure using two subfigures.
% (The subfig.sty package must be loaded for this to work.)
% The subfigure \label commands are set within each subfloat command,
% and the \label for the overall figure must come after \caption.
% \hfil is used as a separator to get equal spacing.
% Watch out that the combined width of all the subfigures on a 
% line do not exceed the text width or a line break will occur.
%
%\begin{figure*}[!t]
%\centering
%\subfloat[Case I]{\includegraphics[width=2.5in]{box}%
%\label{fig_first_case}}
%\hfil
%\subfloat[Case II]{\includegraphics[width=2.5in]{box}%
%\label{fig_second_case}}
%\caption{Simulation results for the network.}
%\label{fig_sim}
%\end{figure*}
%
% Note that often IEEE papers with subfigures do not employ subfigure
% captions (using the optional argument to \subfloat[]), but instead will
% reference/describe all of them (a), (b), etc., within the main caption.
% Be aware that for subfig.sty to generate the (a), (b), etc., subfigure
% labels, the optional argument to \subfloat must be present. If a
% subcaption is not desired, just leave its contents blank,
% e.g., \subfloat[].


% An example of a floating table. Note that, for IEEE style tables, the
% \caption command should come BEFORE the table and, given that table
% captions serve much like titles, are usually capitalized except for words
% such as a, an, and, as, at, but, by, for, in, nor, of, on, or, the, to
% and up, which are usually not capitalized unless they are the first or
% last word of the caption. Table text will default to \footnotesize as
% the IEEE normally uses this smaller font for tables.
% The \label must come after \caption as always.
%
%\begin{table}[!t]
%% increase table row spacing, adjust to taste
%\renewcommand{\arraystretch}{1.3}
% if using array.sty, it might be a good idea to tweak the value of
% \extrarowheight as needed to properly center the text within the cells
%\caption{An Example of a Table}
%\label{table_example}
%\centering
%% Some packages, such as MDW tools, offer better commands for making tables
%% than the plain LaTeX2e tabular which is used here.
%\begin{tabular}{|c||c|}
%\hline
%One & Two\\
%\hline
%Three & Four\\
%\hline
%\end{tabular}
%\end{table}


% Note that the IEEE does not put floats in the very first column
% - or typically anywhere on the first page for that matter. Also,
% in-text middle ("here") positioning is typically not used, but it
% is allowed and encouraged for Computer Society conferences (but
% not Computer Society journals). Most IEEE journals/conferences use
% top floats exclusively. 
% Note that, LaTeX2e, unlike IEEE journals/conferences, places
% footnotes above bottom floats. This can be corrected via the
% \fnbelowfloat command of the stfloats package.




\section{Literature review}
This literature review contains data gathered from the following articles (each of them called later as stated in bold):
\begin{itemize}
\item{EEG-based Emotion Recognition,~2006~\cite{art1} -~\mbox{\textbf{Article~I}}}
\item{EEG Databases for Emotion Recognition,~2013~\cite{art2} -~\mbox{\textbf{Article~II}}}
\item{Evaluating classifiers for Emotion Recognition using EEG,~2013~\cite{art3} -~\mbox{\textbf{Article~III}}}
\item{EEG-based subject independent affective computing models,~2015~\cite{art4} -~\mbox{\textbf{Article~IV}}}
\item{Evaluation of Classifiers for Emotion Detection while Performing Physical and Visual Tasks: Tower of Hanoi and IAPS,~2019~\cite{art5} -~\mbox{\textbf{Article~V}}}
\end{itemize}

\subsection{Performed experiments}

\subsubsection*{Article I}
Protocol of test:
\begin{itemize}
\item{Subject will be instructed about what will happen during test.}
\item{Subject will be exposed to short test run with 3 neutral stimuli.}
\item{During 9 minutes, 36 additional stimuli will be presented, five seconds of stimuli exposure will be followed by 10 seconds of cooling down. Number of test subjects will be limited to five people, with varying gender, age and background.}
\end{itemize}
\vspace{0.in}
User guidelines during the stimulus presentation:
\begin{itemize}
\item{Try not to blink, move your eyes, or move any other part of body.}
\item{Try to stay relaxed and do not tense up.}
\item{Keep your eyes open.}
\end{itemize}
\vspace{0.in}
Instructions during cooling down period:
\begin{itemize}
\item{Blink and move. Relax your shoulders, jaw, neck, face. Do not damage equipment.}
\item{Count down mentally with the numbers presented on the screen.}
\item{Keep your gaze steady on the center cross, where stimulus will happen during final seconds of cooling down.}
\end{itemize}

\subsubsection*{Article II}
In the experiment the Emotiv device had been used. It contained 14 electrodes located in specific points, based on the American Electroencephalographic Society standard. The most important configuration aspects have been:
\begin{itemize}
\item{bandwidth: 0.2-45Hz,}
\item{digital notch filters at 50Hz and 60Hz,}
\item{16-bit AD converters,}
\item{128Hz sampling rate.}
\end{itemize}
Both experiments - audio and visual - had been organized into sessions. For audio experiment, each session consisted consecutively of 12 seconds of silence, followed by 5 to 6 sound clips and ending with time for self-assessment. A session in the visual experiment started with black screen, followed by displaying a cross for concentration purposes.

\subsubsection*{Article III}
There were 20 participants of the experiment. They were asked to watch at 30 emotional related pictures presented sequentially and assess their emotional state (Self-Assessment Manikin) in two dimensions: valence (positive/negative) and arousal (calmness/excitement). In the same time they were examined by EEG. Pictures used in experiment came from IAPS base and were chosen equally from all the emotion cluster (6 for each: neutral, positive arousing/ calm, negative arousing/calm).

\subsubsection*{Article IV}
The group consisted of 26 females. Brain activity was registered by ERP’s recording. Images shown to them were eliting either pleasant or unpleasant emotions. Data itself was taken from IAPS repository (International Affective Picture System). Experiment covered 24 different photos with varying rating (positive and negative). Each picture was presented 3 times for 3500 ms in a pseudo-random order. Signals coming from brain were recorded from 21 channels, sampled at 1kHz.

Raw brain signals were:
\begin{itemize}
\item{filtration using band-pass filter (between 0.1 to 30 Hz),}
\item{eye-movement correction,}
\item{baseline compensation,}
\item{segmentation using epochs using NeuroScan.}
\end{itemize}

\subsubsection*{Article V}
Tower of Hanoi is used because it causes strong emotions to appear. Various systems for emotion classification exist. On of them is Pluchtik’s one, which considers eight primary states – acceptance, anger, anticipation, disgust, fear, joy, sadness and surprise. In this research though, a two dimensional model, involving valence (pleasure) and arousal (excitement), was used. In the experiment, there participated 20 students of Blekinge Institute of Technology in Sweden, aged between 21 and 35 years, coming from different nationalities, cultures and fields of studies. They were asked to solve the Tower of Hanoi puzzle in which they have to move the tower of growing disks from left pig to right, using the middle one as a support. Only one disc can be moved at the time and a bigger disc can never be on top of a smaller one. The subjects did the task twice, once with 4 discs and once with 5 discs, half in this order, and half in the opposite order, to avoid sequencing effect. There was 5 minutes for each of the two tasks.

\subsection{Preprocessing}

\subsubsection*{Article I}
Bandpass filter provided by EEGLab for Matlab was used, fourier frequency analysis signal can be split up in frequencies. Specific frequencies can be removed and the signal can be transformed back, containing only frequencies of interest. At this point alpha and beta bands were available from recorded channels. As it was unknown what combinations would provide the best classification results, all interesting possibilities were tested for channels. Principal component analysis was applied to reduce total number of features from 1000 to 1-25.

\subsubsection*{Article II}
The process of processing the EEG data consisted of several steps. First, the raw data has been filtered by a 2Hz high-pass filter. The first 5 seconds of data, recorded before presenting the stimuli, was considered a baseline. Then a 3-47Hz filter with Welch's method has been applied. Next the baseline power has been subtracted from the data with emotions in order to obtain the relative change of power during the exposure. Then mean changes of power for alpha(8-13Hz), beta(14-29Hz), gamma(30-47Hz) and theta(3-7Hz) signals have been computed. Finally the Spearman correlation coefficients between the power changes and self-assessment ratings for each subject have been calculated. This resulted in the following observations:
\begin{itemize}
\item{For arousal dimension, a negative correlation in theta, beta and gamma bands has been observed,}
\item{For valence dimension there was a positive correlation in gamma band,}
\item{For dominance dimension, there was a positive correlation between it and the beta band of the frontal lobe electrode.}
\end{itemize}

\subsubsection*{Article III}
After collecting data a few results with low emotional ratings were rejected, so data from 15 subjects remained. In the next step signals connected with all the particular emotions were extracted from EEG data, noise, gaps and other artifacts were removed. Finally features were selected according to theirs correlations with emotional ratings. Features were extracted such techniques as Fourier transform, wavelet transform, thresholding, and peak detection and were based on signals from six electrodes and theirs extreme values, mean and standard deviations. Prepared data were used to train classifiers using 10 fold cross validation.

\subsubsection*{Article IV}
Initial step to reduce big variability between subjects was to get averaged signals of all positive and negative trials per subject. The averaged ERP signals were filtered using a Butterworth filter of 4th order with passband [0.5-15] Hz. ERP-based detection systems rely on the frequency content of signals.

\subsubsection*{Article V}
The data was preprocessed using EEGLAB Matlab Toolbox to extract Epoch and Event information, as well as Indepentent Component Analysis. These should help in removing artifacts caused by eye blinking or system impedance, and also make feature extraction easier.

\subsection{Classifiers}
Classifiers used in articles are marked as green in an appropriate cell in table \ref{table_classifiers}.
\definecolor{green}{rgb}{0,0.5,0}
\begin{table}[!t]
\renewcommand{\arraystretch}{1.3}
\caption{Classifiers used in articles}
\label{table_classifiers}
\centering
\setlength\tabcolsep{2pt}
\begin{tabular}{|c|c|c|c|c|c|}
\hline &Article I&Article II&Article III&Article IV&Article V\\\hline
ANN&&&\cellcolor{green}&\cellcolor{green}&\cellcolor{green}\\\hline
LogReg&&&&\cellcolor{green}&\\\hline
LDA&&&&\cellcolor{green}&\\\hline
kNN&&&\cellcolor{green}&\cellcolor{green}&\cellcolor{green}\\\hline
NB&&&&\cellcolor{green}&\\\hline
SVM&&\cellcolor{green}&\cellcolor{green}&\cellcolor{green}&\cellcolor{green}\\\hline
DT&&&&\cellcolor{green}&\\\hline
BN&&&\cellcolor{green}&&\cellcolor{green}\\\hline
FDA&\cellcolor{green}&&&&\\\hline
\end{tabular}
\end{table}

Full names:
\begin{itemize}
\item{Artificial Neural Networks (ANN)}
\item{Logistic Regression (LogReg)}
\item{Linear Discriminant Analysis (LDA)}
\item{k-Nearest Neighbors (kNN)}
\item{Naive Bayes (NB)}
\item{Support Vector Machine (SVM)}
\item{Decision Tree (DT)}
\item{Bayesian Network (BN)}
\item{Fisher’s Discriminant Analysis}
\end{itemize}

\subsection{Conclusions}

\subsubsection*{Article I}
Literature research pointed to alpha and beta frequency bands for emotion recognitions. Various combinations of the two were tested for each of channels. For each of the different classifications (modality, arousal, valence) different features could give the best results. Modality: compared to the results of arousal and valence classifiers, modality classification is apparently more difficult, 82.1\% classification rate. Arousal: 3 errors on 39 classifications, 92.3\% performance. Valence: Performance is good for all channels. Highest performance rate 94.9\%. Binary modality classification rate of over 80\% seems attainable. Visual stimuli appear more difficult to classify than their audio and audiovisual counterparts.

\subsubsection*{Article II}
The special case of happy (positive, high arousal, high dominance) and frightened (negative, high arousal, low dominance) emotions has been studied independently using spatial pattern of Fractal Dimensions. As a result it has been observed, that the right hemisphere of negative emotions, when experiencing fear, is more active than the left one. Moreover this was true for both audio and visual experiment. The data gathered during the experiment has been analyzed using SVM supported by Higher Order Crossing, Fractal Dimensions and statistical classifiers, in different combinations. This method has been applied to both the data collected during the experiment and to the benchmark DEAP database. The aim was to check the accuracy of emotion predictions using different methods and to check if these methods are dependant on stimuli type or measuring device used. It has been concluded that the best results in terms of prediction accuracy are obtained using a combination of HOC, FP and statistical classifiers (87\% for 2 emotions). It has been also observed that a greater amount of measuring channels improves the accuracy of the prediction (90\% for 32 channels and 2 emotions as opposed to 88\% for 16 channels and 2 emotions). An important fact has been that the results were consistent between experimental and benchmark databases, proving that the proposed analysis method is stimuli- and device-independent.

\subsubsection*{Article III}
The best results were obtained for SVM classificator (accuracy 56.10\%) and all the rest results were oscillating around 50\%. After dividing data sets (5 subjects each) to three subsets results were better for two subsets (up to 78\%) and worse for the third. Still the best results were obtained for SVM in every subset, and the second classificator was kNN. According authors the difference in results for whole set and subsets is caused by problem with selecting enough general features, not connected with subjects specific reactions. There were executed one more experiment for single subject subsets and results were up to 83\% (for kNN).

\subsubsection*{Article IV}
The most important task is to select subject specific features. Sequential Feature Selection (SFS) provides the most confident features, but is computationally heavy procedure that may take too long to finish. On the other hand, empirical feature elimination is a compromise between computational time and discrimination performance, because effects can be obtained very short but the they will not be as accurate as in case of SFS. All in all, both methods are consistent in selecting the parietal and occipital channels are better encoders of the class information and less variable across subjects. The topic of emotion valence recognition

\subsubsection*{Article V}
70\% of the subjects considered the Tower of Hanoi problem with 4 disc hard, while 30\% treated them easy. On the other hand, the 5 disc version was evaluated hard by 80\% of subjects and easy by 20\%. SVM proved to be the best at classifying EEG data with specific emotional states with accuracy of 70\%, followed by RT with 60\% accuracy. The study shows that focused brain activities on one goal may yield better accuracy than diverse activities on diverse goals.

\subsection{Summary}
All listed articles covered creating model for performing emotion recognition.
Each article used different classifier (or set of classifiers) for this purpose.
The most distinguishing cases were for Article I and Article II. In Article I only
Fisher's Discriminant Analysis classifier was used. Calculcated performance in terms
of arousal and valance oscilated between 92.3\% and 94.9\%. In Article II data was
analyzed using SVM, giving accuracy around 87\%. It turned out that a greater amount of
channel improves the accuracy of the prediction. In addition results indicated
stimuli- and device-independency, which is an important feature. In contrast, SVM
classifier in Article III offered an accuracy of around 56,10\%. Although it is not
a very high score, among other tested classifiers that was the best method. The best
results for SVM was also obtained in Article V with 70\% accuracy. Only in Article IV
approach using Sequential Feature Selection gave satisfying results but at a cost of
heavy computational power required. 

\section{Materials}

\subsection{Database}
In the database there are results of research conducted on a group of 5 patients. Each patient participated in 3 sessions. During each session, the patient was shown 30 blocks of pictures. Each block constisted of 5 pictures and each picture was shown for about 5 seconds. After each block there was a short break and the patient evaluated their arousal and valence feelings. All pictures in each block were meant to cause similar arousal and valence feelings. After each block, the patient evaluated their valence and arousal feelings. During each session the patient had their EEG recorded.

In the EEG directory, there is a .bdf file for each session of each patient. Each contains data from 72 channels with sampling frequency 1024 Hz, except the first patient, where it is 256 Hz. Also, for each session, there is a .mrk file with numbers of samples at which each block of the session started and finished.

In the Common directory there are files with arousal and valence evaluation of each patient after each picture block. Also, there are files with names and arousal and valence values of each picture used in the research, as well as classes corresponding to each block presented to each patient.

In the fNIRS directory there are fNIRS files corresponding to each session conducted during the research. There is also a MATLAB code that can be used to load this data for further analysis.

\subsection{Data Usefulness Analysis}
During analysis of data some failures occurred while trying to read EEG data for 2nd and 3rd session for first subject, so these data were considered as corrupted. Moreover 1st and 2nd series for first patient have the same assessments and timestamps, and 1st series for second patient has wrong, too high, amount of timestamps, so they both were rejected too. As a result there were used only 11 data series.

Eight last channels of data were rejected because according to database description they contain data coming from other than EEG sensors, e.g. Skin Response or Stimuli Occurance.

In experiment there were used data from 64 EEG channels, marker files and assessment of image blocks. FNIRS data and assessments from IAPS database weren't used.


\section{Methods}
In the experiment data from base were read and extracted from files. Image assessments were transformed into three classes and EEG series were filtered and transformed from time domain to frequency domain. Received data were divided into two sets in 4:1 ratio: training and testing set. Then a neural network was created and trained with training data. Finally the accuracy of the network was measured using test set.

\subsection{Preprocessing}
Preprocessing of EEG data consisted of four parts:
\begin{itemize}
\item{Extracting sections for individual stimuli according to markers}
\item{Filtering data for EEG-useful frequencies, i.e. 1-100~Hz}
\item{Transforming data to frequency domain}
\item{Removing unused channels}
\item{Standardizing data}
\end{itemize}

To extract data from .To extract data from .bdf files and other operation on these files MNE library was used. Marker files were read as text data. Finding and cutting out segments from data was implemented in \texttt{cvapr\_data} module. Transforming data to power spectrum was originally implemented by FFT transform but because of inacceptable time of working finally MNE feature was used. All preprocessing steps were processed in some different orders to experimentally find the best one. Finally they were performed in the following order: extracting sections, removing channels, transforming to power spectrum and standarizing.

For representing EEG data in program, a class \texttt{PictureBlockData} was defined. That class delivered features such as accessing to raw EEG and corresponding assessment, filtering data and transforming it to frequency domain.

Preprocessing of image assessments was significantly easier. Its only step was to transform two dimensional numerical values to three discrete classes: calm, positive and negative.

\subsection{Neural Network Models}
In the experiment there were used convolution neural networks. Some different models were created and tested. All data were provided as 1D data, so networks use only 1D layers. Structures of networks are described below:
\begin{itemize}
\item{Network 1}
\begin{itemize}
\item{Convolution layer with 64 filters of size 11}
\item{Convolution layer with 128 filters of size 7}
\item{Convolution layer with 256 filters of size 3}
\item{Flattening layer}
\item{Full-connected layer with softmax activation function}
\end{itemize}
\item{Network 2}
\begin{itemize}
\item{Convolution layer with 128 filters of size 3}
\item{Flattening layer}
\item{Full-connected layer with sigmoid activation function}
\end{itemize}
\item{Network 3}
\begin{itemize}
\item{Convolution layer with 128 filters of size 3}
\item{Flattening layer}
\item{Full-connected layer with softmax activation function}
\end{itemize}
\item{Network 4}
\begin{itemize}
\item{Convolution layer with 256 filters of size 7}
\item{Flattening layer}
\item{Full-connected layer with softmax activation function}
\end{itemize}
\item{Network 5}
\begin{itemize}
\item{Two convolution layers with 256 filters of size 7}
\item{Flattening layer}
\item{Full-connected layer with softmax activation function}
\end{itemize}
\end{itemize}

Output of all the networks is three-element vector, whose cells correspond to emotion classes. For the all convolution layers the rectified linear unit function was used as activation function. 
As loss function there was used categorical crossentropy and as optimizer - adam. The metric user for measuring network prediction quality was accuracy.

\subsection{Training Networks}
Before training networks test and train data sets were prepared. Train set contained 80\% of all the data, test set was built from remaining 20\%. Both sets were prepared from all used series of data, i.e. every single series was divided in the same ratio. Every model was trained and tested several times because of nondeterministic character of initial weights of network.

Training of network was processed with two stop conditions: target accuracy and no improvements for specified number of iterations. Target accuracy was set to value 0.6 and limit of iterations without improvement to 10. In every iteration training data were shuffled and 5 epochs of training were performed.


\section{Results}
Results of training and testing models are showed in table \ref{table_networks}.
\begin{table}[!t]
\renewcommand{\arraystretch}{1.3}
\caption{Network accuracies}
\label{table_networks}
\centering
%\setlength\tabcolsep{2pt}
\begin{tabular}{|c|c|}
\hline Model&Mean accuracy\\\hline
Network 1&45.57\%\\\hline
Network 2&43.19\%\\\hline
Network 3&47.18\%\\\hline
Network 4&48.27\%\\\hline
Network 5&46.70\%\\\hline
\end{tabular}
\end{table}

Taking into account that 39\% of all data are of class 'calm', results are up to 10 percentage points better than primitive predicting always the most common class, which is moderate success. The reason results weren't higher can be caused by low quality of data. Analysis of data indicated that some data (4 of 15 data series) are corrupted, and we can expect that more detailed researches could reveal abnormalities in other series too. Another significant problem can be labeling, which was based on self-assesments of patients, possibly inaccurate or improper. It is also possible that some contribution in low results is caused by bad understanding of data, not sufficient preprocessing or wrong network structure and network parameters.


\section{Summary}
This project aim was to examine of  effectiveness of using neural network in EEG data analysis. In the experiment there was examined accuracy of emotion recognition based on EEG data. As parts of this tasks there was implemented a convolution neural network and preprocessing module for data, and finally network was trained and its accuracy was measured for test set. Results weren't very high, which can be caused by small amount and low quality of data, but also by wrong parameters or structure of network or low understanding of data. Results suggests that emotion recognition is a complex problem, that requires huge amount of data and efforts to be solved with good results.


% if have a single appendix:
%\appendix[Proof of the Zonklar Equations]
% or
%\appendix  % for no appendix heading
% do not use \section anymore after \appendix, only \section*
% is possibly needed

% use appendices with more than one appendix
% then use \section to start each appendix
% you must declare a \section before using any
% \subsection or using \label (\appendices by itself
% starts a section numbered zero.)
%


\appendices
\section{Author Contributions}
\textbf{Data loading} K. Lelonek, P. Figiel \hfill \break
\textbf{Data processing} P. Figiel, D. Pieła, M. Jakóbczyk \hfill \break
\textbf{Network modelling} D. Pieła, K. Lelonek \hfill \break
\textbf{Testing} P. Figiel, K. Lelonek \hfill \break
\textbf{Literature review} M. Jakóbczyk \hfill \break
\textbf{Data analysis} K. Lelonek \hfill \break
\textbf{README} M. Jakóbczyk \hfill \break
\textbf{Report - contents} K. Janas \hfill \break
\textbf{Report - formatting} K. Janas \hfill \break

\onecolumn
\section{Code}
\subsection*{\texttt{main}}
\lstinputlisting[language=Python, breaklines=true, postbreak=\mbox{{$\hookrightarrow$}\space},]{../main.py}
\subsection*{\texttt{cvapt\_data}}
\lstinputlisting[language=Python, breaklines=true, postbreak=\mbox{{$\hookrightarrow$}\space},]{../cvapr_data.py}
\subsection*{\texttt{prepared\_data\_provider}}
\lstinputlisting[language=Python, breaklines=true, postbreak=\mbox{{$\hookrightarrow$}\space},]{../prepared_data_provider.py}
\subsection*{\texttt{data\_processor}}
\lstinputlisting[language=Python, breaklines=true, postbreak=\mbox{{$\hookrightarrow$}\space},]{../data_processor.py}

% use section* for acknowledgment
%\ifCLASSOPTIONcompsoc
  % The Computer Society usually uses the plural form
%  \section*{Acknowledgments}
%\else
  % regular IEEE prefers the singular form
%  \section*{Acknowledgment}
%\fi


%The authors would like to thank...


% Can use something like this to put references on a page
% by themselves when using endfloat and the captionsoff option.
\ifCLASSOPTIONcaptionsoff
  \newpage
\fi



% trigger a \newpage just before the given reference
% number - used to balance the columns on the last page
% adjust value as needed - may need to be readjusted if
% the document is modified later
%\IEEEtriggeratref{8}
% The "triggered" command can be changed if desired:
%\IEEEtriggercmd{\enlargethispage{-5in}}

% references section

% can use a bibliography generated by BibTeX as a .bbl file
% BibTeX documentation can be easily obtained at:
% http://mirror.ctan.org/biblio/bibtex/contrib/doc/
% The IEEEtran BibTeX style support page is at:
% http://www.michaelshell.org/tex/ieeetran/bibtex/
%\bibliographystyle{IEEEtran}
% argument is your BibTeX string definitions and bibliography database(s)
%\bibliography{IEEEabrv,../bib/paper}
%
% <OR> manually copy in the resultant .bbl file
% set second argument of \begin to the number of references
% (used to reserve space for the reference number labels box)
\begin{thebibliography}{5}

\bibitem{art1}
Danny Oude Bos, \emph{EEG-based Emotion Recognition - The Influence of Visual and Auditory Stimuli}, \hskip 1em plus 0.5em minus 0.4em\relax University of Twente, Enschede, The Netherlands, 2006
\bibitem{art2}
Yisi Liu, Olga Sourina \emph{EEG Databases for Emotion Recognitio}, \hskip 1em plus 0.5em minus 0.4em\relax Nanyang Technological University Singapore, 2013
\bibitem{art3}
Ahmad Tauseef Sohaib, Shahnawaz Qureshi, Johan Hagelback, Olle Hilborn, Petar Jercic \emph{Evaluating classifiers for Emotion Recognition using EEG}, \hskip 1em plus 0.5em minus 0.4em\relax Blekinge Institute of Technology, Karlskrona, Sweden, 2013
\bibitem{art4}
Lachezar Bozhkov, Petia Georgieva, Isabel Santos, Ana Pereira and Carlos Silva \emph{EEG-based subject independent affective computing models}, \hskip 1em plus 0.5em minus 0.4em\relax 2015 INNS Conference on Big Data, 2015
\bibitem{art5}
Shahnawaz Quresh, Johan Hagelback, Syed Muhammad Zeeshan Iqbal, Hamad Javaid and Craig A. Lindley \emph{Evaluation of Classifiers for Emotion Detection while Performing Physical and Visual Tasks: Tower of Hanoi and IAPS}, \hskip 1em plus 0.5em minus 0.4em\relax Intelligent Systems Conference, 2018
\bibitem{db_desc} Arman Savran, Koray Ciftci, Guillame Chanel, Javier Cruz Mota, Luong Hong Viet, BülentSankur, Lale Akarun, Alice Caplier and Michele Rombaut \emph{Emotion Detection in the Loop from Brain Signals and Facial Images}, \hskip 1em plus 0.5em minus 0.4em\relax Proceedings of the eNTERFACE 2006 Workshop, 2006

\end{thebibliography}

% biography section
% 
% If you have an EPS/PDF photo (graphicx package needed) extra braces are
% needed around the contents of the optional argument to biography to prevent
% the LaTeX parser from getting confused when it sees the complicated
% \includegraphics command within an optional argument. (You could create
% your own custom macro containing the \includegraphics command to make things
% simpler here.)
%\begin{IEEEbiography}[{\includegraphics[width=1in,height=1.25in,clip,keepaspectratio]{mshell}}]{Michael Shell}
% or if you just want to reserve a space for a photo:

%\begin{IEEEbiography}{Michael Shell}
%Biography text here.
%\end{IEEEbiography}

% if you will not have a photo at all:
%\begin{IEEEbiographynophoto}{John Doe}
%Biography text here.
%\end{IEEEbiographynophoto}

% insert where needed to balance the two columns on the last page with
% biographies
%\newpage

%\begin{IEEEbiographynophoto}{Jane Doe}
%Biography text here.
%\end{IEEEbiographynophoto}

% You can push biographies down or up by placing
% a \vfill before or after them. The appropriate
% use of \vfill depends on what kind of text is
% on the last page and whether or not the columns
% are being equalized.

%\vfill

% Can be used to pull up biographies so that the bottom of the last one
% is flush with the other column.
%\enlargethispage{-5in}



% that's all folks
\end{document}


